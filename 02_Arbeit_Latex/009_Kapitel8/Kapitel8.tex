%
%% Kapitel: Kapitel 4
%%======================================================================

\chapter{Aussenlinienverfolgung mit dem Line-Follower Algorithmus}
\label{cha:Aussenlinienverfolgung mit dem Line-Follower Algorithmus} \index{Aussenlinienverfolgung mit dem Line-Follower Algorithmus}
%
%

Der Line-Follower-Algorithmus setzt auf den beiden Startpunkt-Algorithmen start of lines search und Vanishin Point search auf. 
Der Algorithmus arbeitet auf dem Graustufenbild in der Birdseye-Ansicht.
Er ben{\"o}tigt einen Startpunkt und einen Startwinkel als Eingabeparameter. 
Ziel ist es eine Au{\ss}enlinie von einem Startpunkt im unteren Bildbereich soweit es m{\"o}glich ist nachzuverfolgen. Das Verfahren ist ein rekursives Verfahren und ruft sich mit neuem Startpunkt und Startwinkel immer wieder selbst auf bis eine von mehreren Terminierungsbedingungen in Kraft tritt.

Angefangen vom ersten Startpunkt betrachtet der Algorithmus eine Anzahl von AR Richtungen mit Sichtweite SW und einem Sichtfeld SF festgelegt durch einen festen Winkel.
Die Mitte des Sichtfeldes wird durch den Startwinkel festgelegt.

[Bild Sichtfeld]

Im ersten Schritt werden alle Pixelintensit{\"a}ten unter den Zweigen der Suchrichtungen in einem Vector-Container gesammelt.
Anschlie{\ss}end wird der Algorithmus Otsus-Methode auf diesen Container angewandt.
Otsus-Methode findet einen Schwellwert der das Bild in zwei Klassen aufteilt.
In diesem Fall der schwarze Hintergrund der Fahrbahn und die wei{\ss}en Linien.
Es wird ein Histogramm aus vorkommenden Pixel-Intensit{\"a}ten erzeugt und ein Schwellwert berechnet, der die gewichtete Interklassen Varianz der beiden Klassen minimiert. 

[Formel]

Mit dem berechneten Otsu-Schwellwert werden die AR Suchrichtungen wieder durchschritten. F{\"u}r jede der Suchrichtungen wird die aktuelle Intensit{\"a}t der Position mit dem Otsu-Schwellwert verglichen. Liegt eine Intensit{\"a}t unter dem Otsu-Schwellwert wird die vorherige Position und die Akkumulation aller Pixel-Intensit{\"a}ten bis zu dieser Position gespeichert. Ansonsten bricht die Suche nach Erreichen von SW in dem derzeitigen Zweig ab und es wird auch hier die Positionen mit akkumulierten Pixel-Intensit{\"a}ten gespeichert.
Sind alle Zweige durchlaufen wird {\"u}ber all jene der nach Intensit{\"a}ten gewichtete Schwerpunkt ermittelt.
[Formel]
Das Pixel im vierer Nachbarschaftsbereich des Schwerpunktes mit h{\"o}chster Intensit{\"a}t wird als neuer Startpunkt festgelegt. Und der neue Suchwinkel ist der Winkel zwischen altem und neuem Startpunkt. Der Algorithmus ruft die Suchfunktion erneut auf bis eines der folgenden Terminierungsbedingungen erf{\"u}llt ist. 

Eine maximale Anzahl von Iterationen wurde erreicht
Die Suche bleibt Stecken
Wenn gefundene Punkte nacheinander sich unter einer festgelegten euklidischen Distanz zum vorhergehenden Punkt befinden wird ein Counter hochgesetzt welcher den rekursiven Aufruf terminiert wenn dieser Counter einen Schwellwert {\"u}berschreitet

Die Suche bewegt sich zur{\"u}ck
Wenn der Y-Wert des vorherigen Startpunktes gr{\"o}{\ss}er ist als der neue Startpunkt wird eine Counter erh{\"o}ht der nach {\"u}berschreiten eines Schwellwertes den Algorithmus terminiert.
Die Suche ist au{\ss}erhalb des Bildbereichs

Ist die Suche beendet gibt line\_follower folgende R{\"u}ckgabeparamter zur{\"u}ck.

Gefundene Linien-Punkte mit zugeh{\"o}rigem Winkel







