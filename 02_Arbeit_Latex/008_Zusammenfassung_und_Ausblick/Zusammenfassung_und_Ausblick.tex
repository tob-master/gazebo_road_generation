%
%% Kapitel: Zusammenfassung und Ausblick
%%======================================================================

\chapter{Zusammenfassung und Ausblick}
\label{cha:Zusammenfassung_und_Ausblick} \index{Zusammenfassung und Ausblick}
%
%
Das Ziel dieser Arbeit war die Entwicklung eines Kalibrierverfahrens, welches die relativen Posen der Sensoren eines Kamera-IMU-Systems sch�tzt. Da diese Art der Systeme aufgrund sinkender Sensorkosten vermehrt als Navigationsl�sung eingesetzt wird, sind in den letzten Jahren einige Kamera-IMU-Kalibrierverfahren vorgestellt worden. Die Abgrenzung zu den vorgestellten Verfahren liegt in den verwendeten Messdaten. W�hrend die bisherigen Verfahren die Kameramessdaten durch Auswertung bekannter Muster erzeugen, sollen in dem neu entwickelten Verfahren VO-Daten eingesetzt werden, welche auf kein bekanntes Muster angewiesen sind. Der Vorteil eines solchen Verfahrens liegt darin, dass die Kalibrierung auch bei Systemen mit einer eingeschr�nkten Kinematik, wie beispielsweise Fahrzeugen, eingesetzt werden kann. F�r die Entwicklung des VO-basierten Kalibrierverfahrens wurden zun�chst bekannte Kalibrierverfahren dahingehend untersucht, ob eine Erweiterung f�r die Verwendung von VO-Daten m�glich ist. Das Ergebnis dieser Untersuchung ergab, dass sich das Kalibrierverfahren nach Furgale et al. \cite{furgale2013unified} am besten eignet. Im Weiteren wurden die Anpassungen der Kalibriermethoden beschrieben und eine softwartechnische L�sung vorgestellt. Mit Hilfe verschieden durchgef�hrter Kalibrierungen sollte die Tauglichkeit dieses Verfahren gepr�ft werden. Dabei konnte durch Messdaten eines simulierten Kamera-IMU-Systems die Wirksamkeit des neu entwickelten VO-basierten Kalibrierverfahrens bewiesen werden. Die Ergebnisse welche durch diese Messdaten erzielt werden konnten, kommen nahe an die Ergebnisse des etablierten Kalibriertools Kalibr (\ref{sec:Kamera-Imu-Kalibrierung mit Kalibr}) heran. Bei der Kalibrierung mit realen Messdaten weicht die Sch�tzung des translatorischen Versatzes jedoch soweit von der tats�chlichen Pose ab, dass dieses Ergebnis keinen praktischen Nutzen erzielt. Die Sch�tzung der Verdrehung dagegen liefert �hnlich gute Ergebnisse wie das Kalibriertool Kalibr und eignet sich daher f�r einen praktischen Einsatz. F�r die Kalibrierung eines realen Systems ergibt sich somit eine Strategie, welche aus einer Kombination der VO-basierten Kalibrierung und dem manuellem Ausmessen des Systems besteht. Die Verdrehung wird dabei durch die VO-basierte Kalibrierung gesch�tzt und der translatorische Versatz durch die manuelle Ausmessung bestimmt.\\
In den Vergleichen des Abschnitts \ref{subsec:DiskussionSimu} konnte gezeigt werden, dass die Abweichung zu Teilen aus Fehlern der VO-basierten Kalibriermethode stammt. Ein anderer Teil stammt aus den fehlerbehafteten Messdaten der IMU und der VO. Im Folgenden werden Punkte aufgef�hrt, welche m�gliche Fehlerquellen bei der Kalibrierung darstellen, und �berlegungen angestellt wie diese Fehler reduziert werden k�nnten:
\begin{itemize}
\item In Abschnitt \ref{subsec:Aufbau der angepassten Software} wurde auf die Vernachl�ssigung der Winkelgeschwindigkeits�nderung hingewiesen, welche sich auf die Genauigkeit des Fehlerterms $e_{a^i_k}$ auswirkt. Durch eine softwaretechnische L�sung kann die Winkelbeschleunigung berechnet und die Genauigkeit des Fehlerterms $e_{a^i_k}$ erh�ht werden.
\item Die Gewichtungen der IMU-Fehlerterme $e_{\omega_k^i}$ und $e_{a^i_k}$ (Gl. \ref{eq:Stefan_Omega_Fehler_a} und \ref{eq:Fehlerterm_Beschl_a}) k�nnen durch die bekannte Genauigkeit der IMU-Sensoren ermittelt werden. Bei den VO-Daten gibt es bisher keine Informationen zu den Genauigkeiten der Messungen. Daher werden f�r die Gewichtungen der VO-Fehlerterme (\ref{eq:VOFehlerOmega} bis \ref{eq:VOFehlerAng}) empirische Werte angenommen. Der Beschreibung des verwendeten VO-Tools libviso2 ist zu entnehmen, dass eine Genauigkeitsbestimmung jeder Messung vorgesehen ist. Wird diese Erg�nzung in libviso2 umgesetzt, kann f�r jeden gebildeten VO-Fehlerterm eine individuelle Gewichtung verwendet werden.
\item Die Analyse der VO-Daten, welche durch libviso2 erzeugt werden, hat teilweise hohe Messungenauigkeiten ergeben. Die Verwendung eines anderen Softwaretools k�nnte genauere Messdaten generieren.
\item Bei der Messdatengenerierung mit der Tr�gerplattform IOSB.amp Q1 standen eine Menge Sensoren zur Verf�gung, welche f�r eine Bewegungssch�tzung eingesetzt werden k�nnen. Bei Systemen dieser Art ist es durch Datenfusion m�glich, Odometriedaten zu erzeugen, welche genauer sind als die Messungen jedes einzelnen Sensors. Bei der Kalibrierung eines solchen Systems k�nnen diese Odometriedaten mit einbezogen werden. 
\end{itemize}
Durch die genannten Punkte k�nnen bei der Kalibrierung Fehlerquellen ausgeschlossen bzw. die Genauigkeit der Messdaten erh�ht werden. Eine Untersuchung dieser Punkte kann somit zu einer Steigerung der Genauigkeit des VO-basierten Kalibrierverfahrens verhelfen und wom�glich auch den translatorischen Versatz in einem Bereich sch�tzen, welcher f�r die praktische Anwendung geeignet ist.



%----------------------------------------------------------------