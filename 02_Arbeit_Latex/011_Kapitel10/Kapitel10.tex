%
%% Kapitel: Kapitel 4
%%======================================================================

\chapter{Mittelliniensuche und Grapherzeugung}
\label{cha:Mittelliniensuche und Grapherzeugung} \index{Mittelliniensuche und Grapherzeugung}
%
%
Um eine robuste Linienerkennung zu erm{\"o}glichen muss auch die Mittellinie in betracht gezogen werden. Der Algorithmus nutzt dabei einen Clustering-Ansatz. Zuerst werden einzelne Pixel, die zu einer Mittellinie geh{\"o}ren in einer Gruppe zusammengef{\"u}hrt. Anschlie{\ss}end werden aufeinander Folgende Mittellinien-Gruppen, wenn sie unter einem gewissen Abstand voneinander entfernt sind, zu Knoten in einem Graphen vereint. In folgendem wird dieser Algorithmus n{\"a}her beschrieben.

Der Algorithmus startet mit dem Einlesen des Graustufenbilds in Birdseye-Perspektive.
Das Eingangsbild wird in Raster der H{\"o}he RH und der Breite RB aufgeteilt. Die einzelnen Bereiche stehen f{\"u}r Gruppen durch welche Mittellinien gekennzeichnet sind. Jeder Bildbereich wird dadurch mit einen Bereichsschl{\"u}ssel gekennzeichnet. Der Schl{\"u}ssel ist ein Tupel definiert durch pair<int,int>. Die einzelnen Bereichsschl{\"u}ssel sind abh{\"a}nig von der Bilddimension und der Rasterisierung.
Der Schl{\"u}ssel eines Bereichs wird beschrieben durch Bildh{\"o}he/RH (erster Wert des Tupels) und Bildbreite/RB (zweiter Wert des Tupels).

[BILD Rasteriesierung]     

Durch eine For-Schleife wird durch das Bild iteriert und die Pixelintensit{\"a}ten PI werden mit dem festgelegten Schwellwert PIS verglichen. Wird ein PI gr{\"o}{\ss}er PIS auf einer Pixelposition PP gefunden so wird die Funktion radial\_scan aufgerufen. Die Funktion sucht in einem radius SR alle PI um PP ab. Wird eine PI gefunden die gr{\"o}{\ss}er als ein maximaler radial Schwellwert MRS ist, dann bricht der Algorithmus ab und geht zur n{\"a}chsten PP {\"u}ber. Falls keiner der PI um die PP gr{\"o}{\ss}er dem MRS ist wird diese PP mit seinem zugeh{\"o}rigen Bereichsschl{\"u}ssel in dem Container multimap<pair<int,int>,Point> gespeichert. Dementsprechend wird das ganze Bild durchlaufen.

[Bild Radial Scan Rasterschl{\"u}ssel]

Im n{\"a}chsten Schritt wird die Anzahl von PP f{\"u}r jeden Bereichsschl{\"u}ssel im Multimap-Container gez{\"a}hlt. Liegt diese Anzahl unter einem minimalen Schwellwert MPPS so wird der Speicherbereich (PPs mit zugeh{\"o}rigem Bereichsschl{\"u}ssel) aussortiert. Die {\"u}brigbleibenden Daten innerhalb des Multimap-Containers k{\"o}nnen als valide Gruppen von Mittellinienpunkte MP angesehen werden. Falls der Multimap-Container leer ist terminiert der Algorithmus und keine Mittellinien wurden gefunden.

Ist die Mittelliniensuche nicht Terminiert beginnt der Aufbau eines ungerichteten Graphen.
Im ersten Schritt werden daf{\"u}r die geometrischen Schwerpunkte GS durch die MP f{\"u}r jeden Bereichsschl{\"u}ssel berechnet.

[Formel]

Anschlie{\ss}end wird die euklidische Distanz zwischen den GS der Bereichsschl{\"u}ssel mit einem Schwellwert verglichen. Ist die Distanz zweier GS unter dem Schwellwert geh{\"o}ren die Bereichsschl{\"u}ssel I und J zusammen und diese Verbindung wird in einem Vector-Container hinterlegt.  
Alle m{\"o}glichen Permutationen werden so verglichen. Durch die Funktion isPermuted wird w{\"a}hrenddessen sichergestellt das nicht dieselben Bereichsschl{\"u}ssel verglichen werden als auch das keine Bereichsschl{\"u}ssel mehrmals mit dem Schwellwert verglichen werden.
Wurden alle Verbindungen von Bereichsschl{\"u}sseln berechnet, werden diese durch den Algorithmus Depth First Search aufgel{\"o}st. Depth First Search ist ein rekursiver Algorithmus der angewandt wird um  Knoten in einem Graphen zu suchen . Im Gegensatz zur Breitensuche BFS wird bei der Tiefensuche DFS zun{\"a}chst ein Pfad vollst{\"a}ndig in die Tiefe beschritten, bevor abzweigende Pfade beschritten werden.


DFS wurde gew{\"a}hlt da die Bereichsschl{\"u}ssel so gut wie immer nur einen Nachbarknoten besitzen und nicht in die Breite gesucht werden muss. Dies ist damit zu begr{\"u}nden das die Mittellinien der Fahrbahn aufeinanderfolgen. Au{\ss}erdem ben{\"o}tigt DFS weniger Speicherressourcen da nicht alle Kind-Knoten eines aktuell zu pr{\"u}fenden Knotens gespeichert werden m{\"u}ssen. Zudem sind die Knoten nach Beendigung der Suche bereits topologisch sortiert. Durch die implementierte Klasse depth\_first\_search werden die Bereichsschl{\"u}ssel zu Graphen zugeordnet. Dabei k{\"o}nnen auch Graphen mit nur einem Knoten entstehen.

Bild[DFS Cluster]

Ist DFS terminiert werden die in den Graphen enthaltenen Bereichsschl{\"u}ssel durch den Abstand ihres GS zum Fahrzeug sortiert. Dies ist durch die Funktion sort() realisiert welche durch {\"U}berladung mit einer Lambda-Funktion nach der kleinsten euklidischen Distanz absteigend zum Fahrzeug hin sortiert. Werden keine Graphen mit einer Anzahl Knoten gr{\"o}{\ss}er eins gefunden, terminiert der Algorithmus. Ist dies nicht der Fall, werden zus{\"a}tzlich der Richtungswinkel und der Abstand eines Knoten zu seinem n{\"a}chsten Nachbarn berechnet.

Vor dem Beenden der Mittellinien Suche werden alle R{\"u}ckgabeparameter gesetzt. Die sind :



%https://stackoverflow.com/questions/3332947/when-is-it-practical-to-use-depth-first-search-dfs-vs-breadth-first-search-bf
%Comparing BFS and DFS, the big advantage of DFS is that it has much lower memory requirements than BFS, because it’s not necessary to %store all of the child pointers at each level.














