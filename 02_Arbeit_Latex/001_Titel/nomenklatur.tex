%==============================================================
%
% Nomenklatur
%
%==============================================================
\chapter*{Nomenklatur}%
  \label{Nomenklatur}%
   % Die folgende Zeile erzwingt einen Eintag ins Inhaltsverz.
  \addcontentsline{toc}{chapter}{Nomenklatur}%
%-----------
\manualmark
\markright{Nomenklatur}
\markleft{Nomenklatur}
%\ohead[]{\headmark}
%\begin{longtable}{p{3cm} p{2cm} p{\textwidth}}  % ACHTUNG Breite
\begin{longtable}{p{0.15\textwidth} p{0.15\textwidth} p{0.55\textwidth}}  % ACHTUNG Breite 0.9
%-----------------------------------------------

\multicolumn{3}{l}{%
\textbf{\textsf{\large Abk�rzungen}}
}\\
%---------------------------------------------


KNN       & & K�nstliches Neuronales Netz\\
DC        & & Gleichstrom\\
PWM       & & Pulsweitenmodulation \\
SSH       & & Secure Shell\\
IC		  & & Integrierter Schaltkreis\\
ROS       & & Robot Operating System\\
CSI       & & Camera Serial Interface\\
MLP       & & Multilayer Perceptron\\
ROI		  & & Region of Interest\\
RAM       & & Random Access Memory\\
GPU       & & Graphics processing unit\\
HOG       & & Histogram Of Oriented Gradients\\
SIFT      & & Scale Invariant Feature Transform\\
SURF      & & Speeded Up Robust Features\\
SVM       & & Support Vector Machine\\
CNN       & & Convolutional Neural Network\\
ILSVRC    & & Large Scale Visual Recognition Challenge\\
CNTK      & & Computational Network Toolkit\\
cuDNN     & & NVIDIA CUDA Deep Neural Network library\\
GTSRB     & & The German Traffic Sign Recognition Benchmark\\
GTSDB     & & The German Traffic Sign Detection Benchmark\\
RCNN      & & Regions with CNN features\\
FPS       & & Frames per second


\end{longtable}

%\cleardoublepage




















