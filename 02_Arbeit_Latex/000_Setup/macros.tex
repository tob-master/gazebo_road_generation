%
% Makros       
% 120409ar
% 
%%=============================================================================
%% Abk�rzungen (Abteilung "Faulheit")
%%=============================================================================
%
% --- Gleichungen 
% Syntax: \beq{NAME DER GLEICHUNG} 
%         \eeq
% Referenz: \ref{eqt:NAME DER GLEICHUNG}
%
% begin
\newcommand{\beq}[1]
           {
            \begin{equation}
            \label{#1}
           }
% end
\newcommand{\eeq}
           {
             \end{equation}
           }
\newcommand{\ba}{\begin{array}}
\newcommand{\ea}{\end{array}}

\newcommand{\bdm}{\begin{displaymath}}
\newcommand{\edm}{\end{displaymath}}
	   
% --- Itemize 
\newcommand{\bi}{\begin{itemize}}
\newcommand{\ei}{\end{itemize}}

% --- Enumerate
\newcommand{\be}{\begin{enumerate}}
\newcommand{\ee}{\end{enumerate}}

% ---
\newcommand{\bd}{\begin{description}}
\newcommand{\ed}{\end{description}}

%R�misches Zahlen
\newcommand{\RNum}[1]{\uppercase\expandafter{\romannumeral #1\relax}}

% Fussnoten
% Syntax: \fn legt fest, wo das Fu�notenzeichen steht
%         \fnt{FUSSNOTENTEXT} legt den Text fest
\newcommand{\fn}{\footnotemark}
\newcommand{\fnt}[1]{\footnotetext{#1}}

\newcommand{\s}{\scriptscriptstyle}
\newcommand{\D}{\displaystyle}

\newcommand{\bff}[1]{\noindent {\textbf{#1}}}

\newcommand{\ol}{\ddot{O}l}
\newcommand{\p}{\partial}
\newcommand{\Dp}{\Delta p}
\newcommand{\te}{$\vartheta$}             % theta
\newcommand{\cT}{\vartheta}               %      Celsiustemperatur muss erhalten bleiben
\newcommand{\Ct}{\vartheta}               %var   Celsiustemperatur kann evtl in t ge�ndert werden

\newcommand{\R}{{\em\bf R}}               % Universelle Gaskonstante fett
\newcommand{\C}{$^\circ$C}                % Grad Celsius
%\newcommand{\C}{~\textcentigrade{}}                % Grad Celsius
%\newcommand{\CC}{^\circ\mbox{C}}                % Grad Celsius
\newcommand{\CC}{\,^\circ\mathrm{C}}                % Grad Celsius
\newcommand{\CCe}{^\circ\mathrm{C}}                % Grad Celsius
\newcommand{\mue}{\textmu}
\renewcommand{\d}{\partial\mspace{2mu}}   % partielles Diff. Zeichen 
\newcommand{\td}{\,\mathrm{d}}           	% totales Diff (d, nicht kursiv)
\newcommand{\ddt}[1]{\frac{\td #1}{\td t}}% zweifach 

%% ggfl. umschreiben mit \text{}
\newcommand{\idx}[1]{_\mathrm{#1}}        % nicht kursiver Index in Gleichungen geht nur mit Umlauten wie "a
\newcommand{\idxi}[2]{_{\mathrm{#1,}{#2}}}% nicht kursiver Index in Gleichungen geht nur mit Index 
\newcommand{\idy}[1]{^\mathrm{#1}}        % nicht kursiver Index in Gleichungen geht nur mit Umlauten wie "a


\newcommand{\bul}{$\bullet$}              % Mark in Tabs
\newcommand{\Q}{$\bullet$}                % mark2 in Tabs
\newcommand{\mc}{\multicolumn}

%\def\dbar{{\mathchar'26\mkern-12mu d}}  %(The space after the \mu" is optional but is added f
\def\dbar{{\mathchar'26\mkern-11mu\mathrm{d}}}  %130102ar hier f�r lmodern,  achtung passt nich bei jedem Font !
%\def\dbar{{\mathchar'26\mkern-9mu\mathrm{d}}}  %130102ar hier f�r times !

\def\bzw{bzw.\ }
\def\bspw{bspw.\ }
\def\ca{ca. }
\def\dh{d.\,h.\ }
\def\etc{etc.\ }
\def\evtl{evtl. }
\def\ggf{ggf.\ }
\def\inkl{inkl.\ }
\def\o�{o.\,�.\ }
\def\og{o.\,g.\ }
\def\so{s.\,o.}
\def\su{s.\,u.}
\def\ua{u.\,a.\ }
\def\zB{z.\,B.\ }
\def\zT{z.\,T.\ }

%% Spezialit�ten f�r klimatechnik
\def\mp{\dot{m}}
\def\m.{\dot{m}} % geht
\def\V.{\dot{V}} % geht
\def\pws{p\idx{W}\idy{s}(T)} % geht

\newcommand{\name}[1]{\textsc{#1}}  % f�r Firmen, Autoren

\def\Re{\mathin{Re}}
\def\Nu{\mathin{Nu}}
\def\Pr{\mathin{Pr}}

%%
%%
%%
\newcommand{\uu}[1]{\emph{#1}}  %Texthervorhebung
%%
%% Refrenzen -------------------------------------------------
%%
%\newcommand{\RefTab}[1]{$\underline{\mbox{Tab.~\ref{#1}}}$}   % 1. Tabellenref. im Text
%\newcommand{\RefFig}[1]{$\underline{\mbox{Abb.~\ref{#1}}}$}   % 1. Bildref.
%\newcommand{\RefTab}[1]{\textbf{Tabelle~\ref{#1}}}             % 1. Tabellenref. im Text
\newcommand{\RefTab}[1]{{\small \color{LinkColor}$\blacktriangleright${Tabelle~\ref{#1}}}} % 1.

%\newcommand{\RefFig}[1]{{\color{blue}$\blacktriangleright$\textbf{Bild~\ref{#1}}}} % 1.
%\newcommand{\RefFig}[1]{{\color{LinkColor}$\blacktriangleright$\textbf{Bild~\ref{#1}}}} % 1.
%\newcommand{\RefFig}[1]{\textbf{Bild~\ref{#1}}}                % 1. Bildref.
%\newcommand{\RefFig}[1]{{\color{LinkColor}$\blacktriangleright${Bild~\ref{#1}}}} % 1.
\newcommand{\RefFig}[1]{{\small \color{LinkColor}$\blacktriangleright${Bild~\ref{#1}}}} % 1.

%\newcommand{\RefTabc}[1]{Tabelle~\ref{#1}}                     % 2. Tabellenref. im Text
\newcommand{\RefTabc}[1]{{\small Tabelle~\ref{#1}}}             % 2.
%\newcommand{\RefFigc}[1]{Bild~\ref{#1}}                        % 2. Bildref.
\newcommand{\RefFigc}[1]{{\small Bild~\ref{#1}}}                % 2. Bildref.

\newcommand{\RefEq}[1]{\mbox{Gl.~(\ref{#1})}}                  % Gleichungen

% \autoref{label} % Erzeugt link inkl dem Wort Abbbildung
% \nameref{label} % Kapiteltext
 
%
%
%
\newcommand{\ZmE}[2]{$#1$~{#2}}                                % Zahl:#1 mit Einheiten#2 #3
%\newcommand{\zme}[2]{$#1$~{#2}}                               % Zahl:#1 mit Einheiten#2 #3
\newcommand{\zme}[2]{#1~\mbox{#2}}                             % Zahl:#1 mit Einheiten#2 #3
\newcommand{\EH}[1]{#1}                                        %  Einheiten
\newcommand{\B}[1]{\mbox{#1}}

%-------------------------------------------------------------
\newfont{\ssf}{cmss10 scaled 1000}
\newfont{\ssb}{cmssbx10 scaled 1000}
\newcommand{\cf}{\ssf}                 % CaptionFonts: in Bildunter- ,Tab?berschriften
\newcommand{\rf}{\em}                  % RefFonts    : Kennzeichnung von Referenzen im Text
\newcommand{\eng}{\tt}                 % englische Begriffe im deutschen Text 

% Hinweis
% Syntax: \oops{�BERSCHRIFT}{TEXT}
%        : nach �berschtift wird automatisch eingef�gt
\newcommand{\oops}[2]{\begin{quote}\textbf{#1}:\\ {#2} \end{quote} }

%---------------------------------------------------------------
% Bildunter- und Tabellen�berschriften
%---------------------------------------------------------------
% z.B. andere Schrift, oder auch Schriftform und andere Abk�rzung
%
% alternaiv : "Abbildungen" lassen und
%  Zeilenumbruch bei Bildbeschreibungen einf�hren \setcapindent{1em}
%
% bessere Methoden: siehe Komaskript
\def\figurename{Bild}          % oder: Bild z.B. {\bfseries Abb.}
%\def\tablename{Tab.}          % oder: z.B. Tafel, Tab.

\renewcommand*{\captionformat}{.~} % DIN l�sst gr�ssen
%\addtokomafont{caption}{\sffamily\small\raggedright}  % kleinere Schrift, linksb�ndig
\addtokomafont{caption}{\sffamily\small}  % kleinere Schrift
%\setkomafont{descriptionlabel}{\sffamily\small}
\setkomafont{captionlabel}{\sffamily\bfseries}
\setcaphanging
\setcapindent{0em}             % kein Einzug

%---------------------------------------------------------------
% Kommentare 
\newcommand{\Kommentar}[1]{{\em #1}} 


% Verstecktes
% Alles innerhalb von \Hide{} oder \ignore{}
% wird von LaTeX komplett ignoriert (wie ein Kommentar)
%------------------------
% methode 1
\newcommand{\Hide}[1]{}
\let\ignore\Hide
\newcommand{\HideB}[1]{}
\let\ignore\HideB

%--eigner Befehl, l�sst sich besser modifizieren
%
\newcommand{\mynewpage}{\newpage}
\newcommand{\myclearpage}{\clearpage}


%\usepackage{tabularx} % automatische Spaltenbreite 
\newcolumntype{L}[1]{>{\raggedright\arraybackslash}p{#1}}
\newcolumntype{C}[1]{>{\centering\arraybackslash}p{#1}}
\newcolumntype{R}[1]{>{\raggedleft\arraybackslash}p{#1}}

\newcommand{\LCIRC}[1]{{\large\textcircled{\normalsize \texttt{#1}}}}
\newcommand{\NCIRC}[1]{\textcircled{ {\small \texttt{#1}} } }


%Umgebung
\def\d@nger{\marginpar[\hfill\dbend]{\dbend\hfill}}
\newenvironment{danger}{\medskip\hspace{0pt}\d@nger}{\medskip}

% for multicolumn
\newcommand{\cg}{\centering}  %120409ar

%%%%%%%%%%%%%%%%%%%%%%%%%%%%%%%%%%%%%%%%%%%%%%%%%%%%%%%%%%%%%%%%%%%%%%%%%%%%%%%%%%%%%%%%%%%%%%%%%%%%%%%%%%%%%%%%%%%%%%%%%%%%%%%%%%%%%%%%%%%%%
%%%%%%%%%%%%%%%%%%%%%%%%%%%%%%%%%%%%%%%%%%%%%%%%%%%%%%%%%%%%%%%%%%%%%%%%%%%%%%%%%%%%%%%%%

\def\Sum{\scalebox{0.7}{$\sum$}}
\def\IntIm{\scalebox{1.2}{$I�}}
