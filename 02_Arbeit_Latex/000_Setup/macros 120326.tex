%
% Makros       
%******************************************************************************

%%=============================================================================
%% Neue, neudefinierte Befehle
%%=============================================================================

\newenvironment{smethode}%
  {\begin{list}{}{\let\makelabel\smethodelabel
  \setlength\labelwidth{10pt}%
  \setlength\leftmargin{\labelwidth+\labelsep}}}%
%  \setlength\leftmargin{-\labelwidth-\labelsep}}}%
  {\end{list}}
  \newcommand*\smethodelabel[1]{\sffamily\bfseries{\color{blue}#1:}\hfil}


\newcommand{\mymarginline}[1]{\marginline{\sffamily\bfseries{#1}}} %Farbe geht nicht
\newcommand{\mymarginpar}[1]%
%           {\marginpar{\sffamily\small\bfseries{#1}}}  % immer linksb�ndig Farbe geht nicht
{\marginpar[\raggedleft{ \sffamily\small\bfseries{#1}}]{\raggedright{\sffamily\small\bfseries{#1}}}}
            %{%
						%\marginpar[\raggedleft{ \sffamily\small\bfseries{#1}}]%
                      %{\raggedright{\sffamily\small\bfseries{#1}}}%
						%}%

           
\newcommand{\todo}[1]{\marginpar{
%    \fcolorbox{red}{white}{\parbox{\marginparwidth}{ 
    \fcolorbox{red}{white}{\parbox{30mm}{ 
            \color{red}\sffamily\small TODO: #1 }}}}
						
\newcommand{\XX}[1]{\marginpar{\color{red} #1}}

% Achtung
\newcommand{\Achtung}{\marginline{\raisebox{-\height}{\includegraphics[width=5mm]{Achtung.png}} }}

\newcommand{\marginlineA}[1]{\marginpar[\raggedleft{\small {\color{LinkColor}\leftpointright} #1}]{\raggedright {\color{LinkColor}\rightpointleft} \small #1}}


%\newcommand{\Danger}[1]{\marginpar[\hfill\dbend #1]{#1 \dbend\hfill}}
\newcommand{\Danger}[1]{\marginpar[\dbend #1 ]{\dbend #1}}


\newcommand\marginlineBend[1]{\marginpar[\raggedleft {\raisebox{+1.4\height} {\color{LinkColor}\dbend} \small #1}]{\raggedright  {\raisebox{+1.4\height} {\color{LinkColor}\dbend} \small #1}} }

%\newcommand\marginlineBendB[1]{\marginpar[\raggedleft {{\lettrine{\large\dbend}{}}}  \small #1]{\raggedright {{\lettrine{\large\dbend}{}} \small #1} }}
%\newcommand\marginlineBendB[1]{\marginpar[\raggedleft {{\lettrine{\normalsize\dbend}{\hspace{0mm}}}}  \small #1]{\raggedright {{\lettrine{\normalsize\dbend}{}}} \small #1} }}
%\newcommand\marginlineBendB[1]{\marginpar[\raggedleft {\makebox[5mm][l]{\dbend}}  \small #1]{\raggedright {{\dbend}} \small #1} }}


%-------------------------------------------------------------------
% Abbildungen **** Empfehlung: Kopiervorlage nutzen und anpassen, nicht Makro!
%-------------------------------------------------------------------
% Vorsicht: Dateinamen in UNIX-Syntax und relativ zu main.tex
%
% Syntax gilt f�r alle unterst�tzten Grafikformate!
% Voraussetzung: die Datei existiert und die Endung wird NICHT 
% im Dateinamen angegeben!
%
% Syntax: 
% \fg {DATEINAME}% = caption
%          {Ort auf Blatt} htbf
%          {BREITE} % vorzugsweise als Teil der TExtbreite z.B. '0.8\textwidth'
%          {ROTATIONSWINKEL}%
%          {Bildtitel im Abbildungsverzeichnis}
%          {BILDUNTERSCHRIFT}%
%
% Kopiervorlage 
% \fg{Datei}{Ort}{Breite}{Rotation}{Bildtitel im Abbildungsverzeichnis}{Bildunterschrift}%

%\newcommand{\fg}[6]
%             {
%              \begin{figure}[#1]
%              \begin{center}
%              \includegraphics[width=#3, angle=#4]{#2}
%              \caption[#4]{%
%              \label{fig:#2}%
%              #5}
%              \end{center}
%              \end{figure}
%             }

             
%-------------------------------------------------------------------
% Abbildungen caption beside
%-------------------------------------------------------------------
% Kopiervorlage: in den Quelltext kopieren, dann dort umbrechen
%\fgB{Datei}{Ort, !h oder tbh}{Breite}{Rotation}{Bildunterschrift}{i}{Text}
%
% Syntax: 
%\fgB {DATEINAME}
%        {Ort, !h oder tbh}   
%        {BREITE} % vorzugsweise als Teil der TExtbreite z.B. '0.8\textwidth'
%        {ROTATIONSWINKEL}
%        {BILDUNTERSCHRIFT}
%        {Ausrichtung Caption= i oder o}
%        {Text im Abbildungsverzeichnis}
%
% Vorsicht: Dateinamen in UNIX-Syntax und relativ zu main.tex
% evtl ist der Abstand zwischen Text und Bild zu klein
% dann �ndern (siehe Koma skript  

%\newcommand{\fgB}[7]
%             {
%              \begin{figure}[#2]
%              %\setcapindent*{1em}  %h�ngender Einzug ab 2. Zeile
%              \begin{captionbeside}[#7]%
%              {#5}[#6][\linewidth]%
%              %[2em]* %Verschiebung nach aussen
%              \includegraphics[width=#3, angle=#4]{#1}
%              \end{captionbeside}%              
%              \label{fig:#1}
%              \end{figure}
%             }             

%-------------------------------------------------------------------
% Tabellen
%-------------------------------------------------------------------
% Syntax: \bt{SPALTENDEFINITION} - Begin Table
%         \et{LABEL}{TABELLENBEZEICHNUNG}
\newcommand{\bt}[3]
           { 
             \begin{table}[!h]
             \begin{center}
             \caption{\label{tbl:#2}#3}                          
             \begin{tabular}{#1}
           }
\newcommand{\et}
           {
            \end{tabular}
            \end{center}
            \end{table}
           }


%%=============================================================================
%% Abk�rzungen (Abteilung "Faulheit")
%%=============================================================================
%
% --- Gleichungen 
% Syntax: \beq{NAME DER GLEICHUNG} 
%         \eeq
% Referenz: \ref{eqt:NAME DER GLEICHUNG}
%
% begin
\newcommand{\beq}[1]
           {
            \begin{equation}
            \label{#1}
           }
% end
\newcommand{\eeq}
           {
             \end{equation}
           }
\newcommand{\beqa}{\begin{eqnarray}}         
\newcommand{\eeqa}{\end{eqnarray}}

\newcommand{\ba}{\begin{array}}
\newcommand{\ea}{\end{array}}

\newcommand{\bdm}{\begin{displaymath}}
\newcommand{\edm}{\end{displaymath}}
	   
% --- Itemize 
\newcommand{\bi}{\begin{itemize}}
\newcommand{\ei}{\end{itemize}}

% --- Enumerate
\newcommand{\be}{\begin{enumerate}}
\newcommand{\ee}{\end{enumerate}}

% ---
\newcommand{\bd}{\begin{description}}
\newcommand{\ed}{\end{description}}

% Fussnoten
% Syntax: \fn legt fest, wo das Fu�notenzeichen steht
%         \fnt{FUSSNOTENTEXT} legt den Text fest
\newcommand{\fn}{\footnotemark}
\newcommand{\fnt}[1]{\footnotetext{#1}}

\newcommand{\s}{\scriptscriptstyle}
\newcommand{\D}{\displaystyle}

\newcommand{\bff}[1]{\noindent {\textbf{#1}}}

\newcommand{\ol}{\ddot{O}l}
\newcommand{\p}{\partial\mspace{2mu}}
\newcommand{\Dp}{\Delta p}
\newcommand{\te}{$\vartheta$}             % theta
\newcommand{\cT}{\vartheta}               %      Celsiustemperatur muss erhalten bleiben
\newcommand{\Ct}{\vartheta}               %var   Celsiustemperatur kann evtl in t ge�ndert werden

\newcommand{\R}{{\em\bf R}}               % Universelle Gaskonstante fett
\newcommand{\C}{$^\circ$C}                % Grad Celsius
%\newcommand{\C}{~\textcentigrade{}}                % Grad Celsius
\newcommand{\CC}{^\circ \mbox{ C}}                % Grad Celsius
\newcommand{\mue}{\textmu}
\renewcommand{\d}{\partial\mspace{2mu}}   % partielles Diff. Zeichen 
\newcommand{\td}{\,\mathrm{d}}           	% totales Diff (d, nicht kursiv)
\newcommand{\ddt}[1]{\frac{\td #1}{\td t}}% zweifach 

%% ggfl. umschreiben mit \text{}
\newcommand{\idx}[1]{_\mathrm{#1}}        % nicht kursiver Index in Gleichungen geht nur mit Umlauten wie "a
\newcommand{\idxi}[2]{_{\mathrm{#1,}{#2}}}% nicht kursiver Index in Gleichungen geht nur mit Index 
\newcommand{\idy}[1]{^\mathrm{#1}}        % nicht kursiver Index in Gleichungen geht nur mit Umlauten wie "a


\newcommand{\bul}{$\bullet$}              % Mark in Tabs
\newcommand{\Q}{$\bullet$}                % mark2 in Tabs
\newcommand{\mc}{\multicolumn}

\def\dbar{{\mathchar'26\mkern-12mu d}}    %( d mit Strich durch f�r unvollst�ndiges Differential The space after the \mu" is optional but is added f


\def\bzw{bzw.\ }
\def\bspw{bspw.\ }
\def\ca{ca. }
\def\dh{d.\,h.\ }
\def\etc{etc.\ }
\def\evtl{evtl. }
\def\ggf{ggf.\ }
\def\inkl{inkl.\ }
\def\o�{o.\,�.\ }
\def\og{o.\,g.\ }
\def\so{s.\,o.}
\def\su{s.\,u.}
\def\ua{u.\,a.\ }
\def\zB{z.\,B.\ }
\def\zT{z.\,T.\ }

%% Spezialit�ten f�r klimatechnik
\def\mp{\dot{m}}
\def\m.{\dot{m}} % geht
\def\V.{\dot{V}} % geht
\def\pws{p\idx{W}\idy{s}(T)} % geht

\newcommand{\name}[1]{\textsc{#1}}  % f�r Firmen, Autoren

\def\Re{\mathin{Re}}    % eng geschriebene Variablen durch \mathin
\def\Nu{\mathin{Nu}}
\def\Pr{\mathin{Pr}}


\newcommand{\uu}[1]{{\it#1}}  %Texthervorhebung

% Punkt + Komma Abst�nde bei Tausendern/Dezimalzahlen ans dt. anpassen
% besser: paket ziffer einbinden
%\mathcode`,="013B
%\mathcode`.="613A

%%
%% Refrenzen -------------------------------------------------
%%
%\newcommand{\RefTab}[1]{$\underline{\mbox{Tab.~\ref{#1}}}$}   % 1. Tabellenref. im Text
\newcommand{\RefTab}[1]{\textbf{Tabelle~\ref{#1}}}             % 1. Tabellenref. im Text
%\newcommand{\RefTab}[1]{{\small \color{LinkColor}$\blacktriangleright${Tabelle~\ref{#1}}}} % 1.

%\newcommand{\RefFig}[1]{$\underline{\mbox{Abb.~\ref{#1}}}$}   % 1. Bildref.
%\newcommand{\RefFig}[1]{{\color{blue}$\blacktriangleright$\textbf{Bild~\ref{#1}}}} % 1.
%\newcommand{\RefFig}[1]{{\color{LinkColor}$\blacktriangleright$\textbf{Bild~\ref{#1}}}} % 1.
\newcommand{\RefFig}[1]{\textbf{Bild~\ref{#1}}}                % 1. Bildref.
%\newcommand{\RefFig}[1]{{\color{LinkColor}$\blacktriangleright${Bild~\ref{#1}}}} % 1.
%\newcommand{\RefFig}[1]{{\small \color{LinkColor}$\blacktriangleright${Bild~\ref{#1}}}} % 1.

\newcommand{\RefTabc}[1]{Tabelle~\ref{#1}}                     % 2. Tabellenref. im Text
%\newcommand{\RefTabc}[1]{{\small Tabelle~\ref{#1}}}             % 2.
\newcommand{\RefFigc}[1]{Bild~\ref{#1}}                        % 2. Bildref.
%\newcommand{\RefFigc}[1]{{\small Bild~\ref{#1}}}                % 2. Bildref.

\newcommand{\RefEq}[1]{\mbox{Gl.~(\ref{#1})}}                  % Gleichungen

% \autoref{label} % Erzeugt link inkl dem Wort Abbbildung
% \nameref{label} % Kapiteltext
 
%
%
%
\newcommand{\ZmE}[2]{$#1$~{#2}}                                % Zahl:#1 mit Einheiten#2 #3
%\newcommand{\zme}[2]{$#1$~{#2}}                               % Zahl:#1 mit Einheiten#2 #3
\newcommand{\zme}[2]{#1~\mbox{#2}}                             % Zahl:#1 mit Einheiten#2 #3
\newcommand{\EH}[1]{#1}                                        %  Einheiten
\newcommand{\B}[1]{\mbox{#1}}

%-------------------------------------------------------------

\newfont{\ssf}{cmss10 scaled 1000}
\newfont{\ssb}{cmssbx10 scaled 1000}

\newcommand{\cf}{\ssf}                 % CaptionFonts: in Bildunter- ,Tab?berschriften
\newcommand{\rf}{\em}                  % RefFonts    : Kennzeichnung von Referenzen im Text
\newcommand{\eng}{\tt}                 % englische Begriffe im deutschen Text 

% Hinweis
% Syntax: \oops{�BERSCHRIFT}{TEXT}
%        : nach �berschtift wird automatisch eingef�gt
\newcommand{\oops}[2]{\begin{quote}\textbf{#1}:\\ {#2} \end{quote} }

%---------------------------------------------------------------
% - F�r �bungen und Aufgaben
%
% Z�hler f�r �bungen
\newcounter{iexample} 
\setcounter{iexample}{0}
% \addtocounter{iexample}{}
% \stepcounter{iexample}

% f�r Aufgaben
\newcounter{iproblem}
\setcounter{iproblem}{0}
% \addtocounter{iproblem}{}
% \stepcounter{iproblem}

% Hochzahlen und drucken 
\newcommand{\iexap}{\stepcounter{iexample}\arabic{chapter}.\arabic{iexample}}
\newcommand{\iprob}{\stepcounter{iproblem}\arabic{chapter}.\arabic{iproblem}}


%\newcommand{\example}[1]{\noindent {\sffamily\textbf{\itshape\large Beispiel~\iexap:}}~#1\newline}
%\newcommand{\problem}[1]{\noindent {\sffamily\textbf{\itshape\large Aufgabe~\iexap:}}~#1\newline}
%\newcommand{\example}[1]{\noindent\hspace*{0em}{\sffamily\textbf{\itshape\large Beispiel~\iexap:~#1}}\newline\vspace*{-8mm}}
\newcommand{\example}[1]{\noindent\hspace*{-1mm}%
{\sffamily\itshape\textbf{Beispiel~\iexap:}~#1}\newline\vspace*{-4mm}} % 
% -8mm if single

\newcommand{\problem}[1]{\noindent{\sffamily\textbf{\itshape\large Aufgabe~\iexap:~#1}}\newline}

\newcommand{\meth}[1]{\noindent{\color{blue}\sffamily\textbf{#1:}}}

%---------------------------------------------------------------
% Bildunter- und Tabellen�berschriften
%---------------------------------------------------------------
% z.B. andere Schrift, oder auch Schriftform und andere Abk�rzung
%
% alternaiv : "Abbildungen" lassen und
%  Zeilenumbruch bei Bildbeschreibungen einf�hren \setcapindent{1em}
%
% bessere Methoden: siehe Komaskript
\def\figurename{Bild}          % oder: Bild z.B. {\bfseries Abb.}
%\def\tablename{Tab.}          % oder: z.B. Tafel, Tab.

\renewcommand*{\captionformat}{.~} % DIN l�sst gr�ssen
%\addtokomafont{caption}{\sffamily\small\raggedright}  % kleinere Schrift, linksb�ndig
\addtokomafont{caption}{\sffamily\small}  % kleinere Schrift
%\setkomafont{descriptionlabel}{\sffamily\small}
\setkomafont{captionlabel}{\sffamily\bfseries}
\setcaphanging
\setcapindent{0em}             % kein Einzug


%---------------------------------------------------------------
% Literaturliste: Formatierung der Liste wird HIER vorgenommen!
%---------------------------------------------------------------
\newcommand{\lit}[4]
           {
              \bibitem{#1}
                 {#2:}
                 #3 
                 #4 
           }
%---------------------------------------------------------------
%
%---------------------------------------------------------------
% Kommentare 
\newcommand{\Kommentar}[1]{{\em #1}} 

% Verstecktes
% Alles innerhalb von \Hide{} oder \ignore{}
% wird von LaTeX komplett ignoriert (wie ein Kommentar)
%------------------------
% methode 1
\newcommand{\Hide}[1]{}
\let\ignore\Hide
% methode 2
%\newcommand{\Hide}[1]{
%{\color{cyan}\sffamily\small #1}} % green red blue magenta cyan
% methode 3
%\newcommand{\Hide}[1]{
%\let\ignore\Hide #1}

\newcommand{\HideB}[1]{}
\let\ignore\HideB

%--eigner Befehl, l�sst sich besser modifizieren
%
\newcommand{\mynewpage}{\newpage}
\newcommand{\myclearpage}{\clearpage}


%% Links
%
% \url{http://...}
% \href{file#mytarget}{text} 

%\usepackage{tabularx} % automatische Spaltenbreite 
\newcolumntype{L}[1]{>{\raggedright\arraybackslash}p{#1}}
\newcolumntype{C}[1]{>{\centering\arraybackslash}p{#1}}
\newcolumntype{R}[1]{>{\raggedleft\arraybackslash}p{#1}}

\newcommand{\LCIRC}[1]{{\large\textcircled{\normalsize \texttt{#1}}}}
\newcommand{\NCIRC}[1]{\textcircled{ {\small \texttt{#1}} } }

% \marginline{\raisebox{-\height}{\includegraphics[width=2.5cm]{Achtung.png}} und weiterer TExt um mal zu sehn wie es langsam geht  } \\

%Umgebung
\def\d@nger{\marginpar[\hfill\dbend]{\dbend\hfill}}
\newenvironment{danger}{\medskip\hspace{0pt}\d@nger}{\medskip}



