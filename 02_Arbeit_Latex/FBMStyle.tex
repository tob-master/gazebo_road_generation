%%==========================================================================================
%%
%% This is file 'FBMStyle.tex'  v 1.0
%% see also file 'FBMStyle_develop.tex
%%
%% Beta-Version eines Styles von Michael Arnemann
%% basieren auf einem Muster von Uwe Kappler 
%%
%% 2015-03-22 
%%========================================================================================== 
\documentclass[
        final,                % wenn es denn fertig ist.. 
        11pt,                 % Schriftgr��e (10pt, 11pt, 12pt) 11:pala:ok, 
        twoside,              % oneside, twoside
        numbers=noenddot,     % Kapitelnummer ohne . am Ende
        headings=normal,       % Groesse der Ueberschrift (bigheadings, normalheadings, 
        parskip=half,          % Europ�ischer Satz mit Abstand zwischen Abs�tzen
        index=totoc,             % Index ins Verzeichnis einf�gen	*** pr�fen     
%        headsepline,          % Strich unter Kopfzeile
        headinclude,          % Kopfzeile bei Satzspiegel bereucksichtigen
        DIV15,                % mit palatino 11pt oder CM 12
        BCOR14mm]             % Binderand    %1mm, f�r Klebebindung % 14mm dann etwa gleichm��ig
   {scrbook}                  % Dokumenttyp // f�r Dipomarbeiten, Diss
%----------------------------------------------------------------------------------------
%  PACKAGES
%----------------------------------------------------------------------------------------
\typearea                  % nach der Schrift
        [current]          % Heftrand BCOR oben definiert
%        {calc}             % DIV neu berechnen aus package 
        {last}             % DIV letzte aus Definition zuvor

\usepackage{lmodern}          % empfohlen anstatt CM Fonts, nur so korrekte Darstellung mit CM
%\usepackage{mathptmx}
%\fontfamily{ptm}\selectfont  %times

%\usepackage{helvet}
%\fontfamily{phv}\selectfont  %helvetica

\setkomafont{footnote}{\normalsize}                     % Marke und Text einer Fu�note
\usepackage{scrpage2}         % Package laden  

\usepackage[T1]{fontenc}      % T1-encoded fonts: auch W�rter mit Umlauten trennen
\usepackage[T1]{url}          % much like \verb allow line breaks for paths and URLs
\usepackage[latin1]{inputenc} % Deutsche Umlaute: Eingabe nach ISO 8859-1 (Latin1)

\usepackage{epsfig,xspace}    % PS Bilder
\usepackage{graphicx,color}   % JPEG und PNG 
                              % http://en.wikibooks.org/wiki/LaTeX/Colors  
\usepackage{subfigure}	      % Mehrere Bilder in einem mit ver. Bildunterschriften
\usepackage{wrapfig}	        % Text um Bild

\usepackage[ngerman]{babel}   % Neue deutsche Rechtschreibung  120412ar
%\usepackage[babel,german=guillemets]{csquotes} % f�r quotes
\usepackage[babel,german=quotes]{csquotes} % f�r quotes

%tools}             % enth�lt: afterpage, array, bm, calc, dcolumn, delarray, enumerate, fileerr, fontsmpl, ftnright, hhline, indentfirst, layout, longtable, multicol, rawfonts, showkeys, somedefs, tabularx, theorem, trace, varioref, verbatim, xr, and xspace.

\usepackage{latexsym}         % Sonderzeichen  f�r ??
\usepackage{pifont}           % dito
\usepackage{array}            % f�r aufw�ndigere Tabellen
\usepackage{longtable}        % seiten�bergreifende Tabellen passt zu KOMA
\usepackage{multicol}         % Mehrspaltiger Satz

\usepackage{makeidx}          % f�r Index-Erstellung 
%\usepackage{listings}         % f�r Latex Quelltext
%\usepackage{textcomp}         % for upright mu (\textmu)
\usepackage[fleqn]{amsmath}
\usepackage{amssymb}
\usepackage{eqnarray}	        % nummerierte und unnummerierte Gleichungen/systeme
\usepackage{ifthen}
\usepackage{fancybox}         % f�r schattierte ovale Boxen etc. geht nicht mit Miktex 5 060614ar
\usepackage{lscape}           % f�r landscape
\usepackage[normalem]{ulem}   % Unterstreichen und durchstreichen  Probleme mit bibtex
\usepackage{ziffer}           % Komma in math. Umgebung ohne folgendes Leerzeichen
\usepackage{cancel}           % K�rzen in Br�chen 

\usepackage[plainpages=false,pdfpagelabels]{hyperref}         
%
%% **** END OF CLASS MStyle ****