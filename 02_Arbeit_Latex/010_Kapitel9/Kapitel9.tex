%
%% Kapitel: Kapitel 4
%%======================================================================

\chapter{Aufbau eines Testframeworks mit Google-Test}
\label{cha:Aufbau eines Testframeworks mit Google-Test} 
\index{Aufbau eines Testframeworks mit Google-Test}
%
%

Google-Test ist eine Unit-Test-Bibliothek f\"ur die Programmiersprache C++, die auf der xUnit-Architektur basiert. Das xUnit Framework erlaubt das \"Uberpr\"ufen verschiedener Elemente von Software, wie etwa Funktionen und Klassen. Um eine fehlerfreie Software zu gew\"ahrleisten ist ein Unit-Testing der verschiedenen Softwaremodule unumg\"anglich.
Aus diesem Grund wurde die gesamte Bildverarbeitungssoftware modular gestaltet. Jede Funktion der verschiedenen Klassen besitzt definierte Ein- und Ausgangsvariablen und ist intern von keinen weiteren Variablen abh\"angig um die Funktionen separiert testen zu k\"onnen.
F\"ur jede Klasse der in XXX beschriebenen Bildverarbeitungsalgorithmen wurde ein angepasstes Unit-Test-Modul geschrieben, welches getrennt von der Hauptsoftware ausf\"uhrbar ist. Es erlaubt die Initialisierungsparameter anzupassen und die Funktionen mit Mock-Objekten testen zu lassen. Auch k\"onnen die ganzen Klassen auf ihr Verhalten auf Einzelbilder getestet werden. Das Testen wird auch durch eine visualisierte Ausgabe der Verarbeitung des Eingangsbildes erm\"oglicht.










